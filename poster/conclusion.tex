\section{Conclusion and Future Work}\label{sec:conclusion}
We describe VeLoc that can track the vehicle's movements and estimate the final parking location using the smartphone's inertial sensor data only. 
%It does not depend on GPS or WiFi signals which may be be available in environments such as underground parking lots, or require additional sensors to instrument the environment. VeLoc first estimate the pose of the smartphone relative to the vehicle, so that inertial sensor data can be transformed into the coordinate system of the vehicle. It then detects landmarks such as speed bumps, turns and slopes, and combine them with the map information to estimate the vehicle's location using a probabilistic model. 
Experiments in three parking structures have shown that VeLoc can track the parking locations to within 4 parking spaces, which is enough for the driver to trigger a honk using the car key.

Currently VeLoc depends on accurate parking structure maps to reduce the uncertainty in the vehicle location. Since such maps are not always available, we plan to study how to obtain the map information, and track the vehicle when only incomplete and/or inaccurate map is available. This further extend VeLoc's capability in the real world.
%\textbf{How to acquire the map?}
