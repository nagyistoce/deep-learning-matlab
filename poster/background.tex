\section{Related Work}\label{sec:background}
We present a brief introduction of the related work below to distinguish components of Veloc from existing technologies for estimating the phone pose, detecting the landmarks, monitoring the states of a robot, a pedestrian or a vehicle, etc.

\textbf{Phone pose estimation}. It is impractical to assume that the smartphone inside a vehicle has a known position or it is placed stationary. It is necessary to periodically estimate the phone pose in the vehicle. Wang et al. found it possible to align the smartphone's and vehicle's coordinate systems using gravity, accelerometer and gyroscope sensors~\cite{Wang:2013_Driver_Phone_Use}. Specifically, it aligns z-axis of the vehicle with its gravity direction; gyroscope is used to determine whether the vehicle is driving straight, then accelerometer readings are extracted as vehicle driving direction, namely y-axis of the vehicle.
%; at last y-axis is computed by crossing the other two orthometric axes.
However, this approach fails to consider the case when the vehicle runs on a slope, and simply extracting accelerometer readings as vehicle direction is not robust.



\textbf{Virtual landmark detection}.
In robot localization system, a robot is assumed to be exploring the space of interest that has various landmarks (e.g., barcode pasted on walls or a particular pattern painted on the ceiling). The equipped sensors on a robot, such as laser-based ranging and cameras, are used to detect these artificially placed landmarks. However, for smartphone-based indoor localization, it is the smartphone that is carried around by a user to explore the space of interest. It is impossible to have robot sensors (e.g., laser ranging) on a commodity phone. Thus, researchers refer to virtual landmarks which are essentially ambient signatures or recognized patterns/activities that are perceivable by smartphone sensors~\cite{Choudhury:SurroundSense, Acoustic_Background_Spectrum}, and UnLoc \cite{Wang:UnLoc} is believed to be the first to apply virtual landmarks towards deadreckoning. In VeLoc, we use sensor measurements to detect all kinds of road anomaly and turnings as landmarks. In addition, sensor measurements also provide cues for estimating the state of vehicles. For instance, different patterns between a immobile vehicle and a moving one can be view as a measurement of the velocity of the vehicle.

\textbf{Robotic localization}. SLAM is a popular technique in robotics which allows the robot to acquire a map of its environment while simultaneously localizing itself relative to this map~\cite{FastSLAM}. Recently, WiFi-SLAM \cite{WiFi-SLAM} was proposed to utilize the WiFi signal strength as the input of SLAM. Unlike SLAM, VeLoc assumes the availability of a map and the problem to be addressed is equivalent to the robot localization problem of determining the pose of a robot relative to a given map of the environment~\cite{probabilistic}.

The early work on robot localization problem used Kalman Filters which is thought to be the earliest tractable implementations of the Bayes filter for continuous spaces. Subsequent work has been based on Markov localization, which is a better match in practice since it allows the robot's position to be modeled as multi-modal and non-Gaussian probability density functions. Of particular interest to us is the Monte Carlo Localization (MCL), or particle filtering based approach~\cite{MCL, MCLrobust}. Instead of representing the distribution by a parametric form, particle filters represent a distribution by a set of samples drawn from this distribution. Those particles are then evolved based on the action model and the measurements \cite{probabilistic}. Again, robot localization typically depends on using explicit environment sensors, such as laser range finders and cameras. Moreover, the rotation of the robot wheels offer a precise computation of displacement.

Unlike robot localization systems, VeLoc is independent of laser ranging or cameras, but it uses smartphone sensors to compute the displacement and direction of vehicles and detect the virtual landmarks that are only specifically available in parking lots.


\textbf{Dead-reckoning}. Dead reckoning using inertial sensors is a well explored approach to monitor the states of a moving object or a pedestrian. However, conventional sensors used in these applications are very expensive. Recently, it is attractive to use smartphone sensors in indoor environments \cite{Constandache:Did_You_See_Bob}, since consumer mobile devices are increasingly being equipped with sensors such as accelerometer, gyroscope, magnetometer and barometer. However, directly applying this approach in indoor environments is non-trivial since many factors cause fluctuations in acceleration, resulting in erroneous displacements.

Many methods have attempted to mitigate the accumulation of error. Foot-mounted sensors have been shown effective in reducing the error~\cite{Robertson:Foot-mounted_Inertial, Woodman:Pedestrian_Localisation:Foot-mounted}. However, the accumulation of error remains when a smartphone pose is unknown. Outdoor localization schemes like CompAcc~\cite{Choudhury:CompAcc} employ a periodic GPS measurement to recalibrate the user��s location. UnLoc~\cite{Wang:UnLoc} provides an option to replace GPS with virtual indoor landmarks that can be detected using existing sensing modalities for calibration. Dead-reckoning techniques have been widely used in mobile computing community with the purpose of addressing the indoor localization problem~\cite{Rai:Zee}.

To prevent the error accumulation, VeLoc simultaneously harnesses constraints imposed by the map and environment sensing. The only external input for VeLoc is a map of the indoor space of interest. Since the map of a place do not change for several months or years, no repeated manual efforts for calibration are required by VeLoc. In addition, the need for special-purpose hardware and infrastructure is avoided to make VeLoc more practical for the real-world use.

\textbf{Estimation of vehicle states}. There have been many active research efforts in using smartphones' embedded sensors (1) to monitor the states of vehicles (e.g. dangerous driving alert~\cite{Lindqvist:Undistracted_Driving}, car speaker~\cite{Yang:Car_Speakers} and CarSafe~\cite{CarSafe}); (2) to inspect the road anomaly/condition (e.g., Pothole Patrol~\cite{Eriksson:The_Pothole_Patrol} and Nericell ~\cite{Mohan:Nericell}); and (3) to detect traffic accidents (Nericell ~\cite{Mohan:Nericell} and WreckWatch~\cite{WreckWatch}).

% $P^2$ \cite{Eriksson:The_Pothole_Patrol} use mobile sensor measurements to detect road anomaly surface monitoring.

The vehicle speed is a critical input for implementing these applications. It is easy to calculate the outdoor vehicle speed by using the phone GPS~\cite{Virtual_Trip_Lines, VTrack}, while the GPS signal is weak or even unavailable at indoor parking lots. Some alternative solutions leverage the phone's signal strength to estimate the vehicle speed~\cite{Vehicular_Speed, signal_profile}.
