\begin{abstract}\label{sec:abstract}
%Remembering where a vehicle was parked has proven a hassle in large parking structures. Existing RF signature based indoor localization technology is not applicable where such signals may not be available such as at underground parking lots.
%Instrumenting additional sensors may solve the problem but at the cost of significant overheads in time, money and human efforts. 

We present VeLoc, a smartphone-based vehicle localization approach that tracks the vehicle's parking location using the embedded accelerometer and gyroscope sensors.
It harnesses constraints imposed by the map and landmarks (e.g., speed bumps) recognized from inertial data, employs a Bayesian filtering framework to estimate the location of the vehicle. 
%It is also robust to different orientations of the phone relative to the vehicle, and even jolting during bumpy rides. 
We have conducted %extensive 
experiments in 3 parking lots of different sizes and structures, using 3 vehicles and 3 kinds of driving styles. We find that VeLoc can always localize the vehicle within $10m$, which is sufficient for the driver to trigger a honk using the car key.
%This is achieved even when the initial position and heading direction of the vehicle is unknown.


%(xx what other performance numbers or nice features can we boast?)


% a in lacking unique features While WiFi-based indoor localization is attractive, there are many indoor places without WiFi coverage but also show strong demand for localization. This paper describes a system and associated algorithms to address the indoor vehicle localization problem without installation of additional infrastructure. In this system, which we call the \textbf{VeLocE}, we utilize the sensor data of smartphones in the vehicle together with the floor map of the parking lot to track the vehicle. VeLocE simultaneously harness constraints imposed by the map and environment sensing. All these cues are codified into a novel Augmented Particle Filtering framework to estimate the position of the vehicle. Experimental results show that VeLocE performs well even the initial position and the initial heading direction of the vehicle is absolutely unknown.
\end{abstract}

